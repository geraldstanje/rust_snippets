\documentclass{beamer} %
\usetheme{CambridgeUS}
\usepackage[latin1]{inputenc}
\usefonttheme{professionalfonts}
\usepackage{times}
\usepackage{tikz}
\usepackage{amsmath}
\usepackage{verbatim}
\usepackage{graphicx}
\usepackage{multimedia}
\usepackage{listings}
%\usepackage{movie15}
%\usepackage{media9}
\usetikzlibrary{arrows,shapes}

\author{Gerald Stanje}
\title{From C++ to Rust}

\title[] %optional
{From C++ to Rust}
 
\subtitle{A high level overview about two Systems Programming Languages}
 
\author[] % (optional, for multiple authors)
{Gerald Stanje}

\begin{document}

\frame{\titlepage}

% For every picture that defines or uses external nodes, you'll have to
% apply the 'remember picture' style. To avoid some typing, we'll apply
% the style to all pictures.
\tikzstyle{every picture}+=[remember picture]

% By default all math in TikZ nodes are set in inline mode. Change this to
% displaystyle so that we don't get small fractions.
\everymath{\displaystyle}

% Slide 1
\begin{frame}
\frametitle{Content}

\tikzstyle{na} = [baseline=-.5ex]

\begin{itemize}
    \item Systems Programming
    \item Memory management in C++
    \item Static Analysis
    \item Rust
    \item Ownership in Rust
    \item Borrow in Rust
    \item Concurrency in Rust
    \item Cargo
\end{itemize}

\end{frame}

% Slide 2
\begin{frame}
\frametitle{System Programming}

\begin{itemize}
    \item Programmer needs very explicit control over the hardware
    \begin{itemize}
    	\item How much memory is used?
	\item What code is generated?
	\item When the memory is allocated or freed?
    \end{itemize}
    \item Used to build: operating systems, compilers, device drivers, factory automation, robots, high performance mathematical software, games
\end{itemize}

\end{frame}

% Slide 3
\begin{frame}
\frametitle{Memory management in C++ is hard}

\movie[externalviewer]{Video}{video.mp4}

\end{frame}

% Slide 4
\begin{frame}[fragile]
\frametitle{C++ Safety}
\lstset{language=C++,
                basicstyle=\ttfamily,
                keywordstyle=\color{blue}\ttfamily,
                stringstyle=\color{red}\ttfamily,
                commentstyle=\color{green}\ttfamily,
                morecomment=[l][\color{magenta}]{\#}
}
\begin{lstlisting}
int main() {
    vector<string> vec;
    vec.push_back("FC");
    auto& elem = vec[0];
    vec.push_back("Bayern Munich");
    cout << elem << endl;
}
\end{lstlisting}

\end{frame}

% Slide 5
\begin{frame}[fragile]
\frametitle{C++ Safety}
\lstset{language=C++,
                basicstyle=\ttfamily,
                keywordstyle=\color{blue}\ttfamily,
                stringstyle=\color{red}\ttfamily,
                commentstyle=\color{green}\ttfamily,
                morecomment=[l][\color{magenta}]{\#}
}
\begin{lstlisting}
int main() {
    vector<string> vec;
    vec.push_back("FC");
    auto& elem = vec[0];
    vec.push_back("Bayern Munich");
    cout << elem << endl;
}
\end{lstlisting}

\begin{figure}
\centering
\includegraphics[scale=0.25]{safety.png}
\end{figure}

\end{frame}

% Slide 6
\begin{frame}
\frametitle{Static Analysis}

\begin{itemize}
    \item Lots of free, open source and commercial offerings for static analysis of C++ source
    \begin{itemize}
	\item cppcheck
	\item clang-analyse
	\item coverity
    \end{itemize}
\item Analyzers for code guideline profiles
    \begin{itemize}
	\item bounds
	\item types
	\item lifetimes
    \end{itemize}
\item Downside: False Positives $\rightarrow$ cannot see if a certain condition can never happen if e.g. it depends on input.
\end{itemize}

\end{frame}

% Slide 7
\begin{frame}
\frametitle{What is Rust?}

\begin{itemize}
    \item systems programming language
    \item blazingly fast
    \item compiled binary
    \item immutable by default
    \item prevents almost all crashes: no segmentation faults, no null pointers, no dangling pointers
    \item eliminates data races: two parallel processes access memory location in parallel, at least one process writes to the memory.
    \item uses LLVM in the backend
\end{itemize}

\end{frame}

% Slide 8
\begin{frame}[fragile]
\frametitle{Ownership in Rust}

\begin{itemize}
    \item Exactly one owner per allocation
    \item Memory freed when the owner leaves scope
    \begin{itemize}
    	\item All references must be out of scope too
    \end{itemize}
    \item Ownership may be transferred (move)
    \begin{itemize}
    	\item Invalidates prior owner
    \end{itemize}
\end{itemize}

\lstset{language=Java,
                basicstyle=\ttfamily,
                keywordstyle=\color{blue}\ttfamily,
                stringstyle=\color{red}\ttfamily,
                commentstyle=\color{green}\ttfamily,
                morecomment=[l][\color{magenta}]{\#}
}
\begin{lstlisting}
fn print(a: Vec<i32>) {}

fn main() {
    let s = Vec::new();
    a.push(1);
    a.push(2);
    
    print(s);
    print(s); // error: s is no longer the 
    // owner of the vector
}    
\end{lstlisting}

\end{frame}

% Slide 9
\begin{frame}[fragile]
\frametitle{Shared Borrow in Rust}

\begin{itemize}
    \item Ownership may be temporary (borrow)
    \begin{itemize}
    	\item References are created with \&
    \end{itemize}
    \item Borrowing prevents moving
    \item Shared references are immutable
\end{itemize}
    
\lstset{language=Java,
                basicstyle=\ttfamily,
                keywordstyle=\color{blue}\ttfamily,
                stringstyle=\color{red}\ttfamily,
                commentstyle=\color{green}\ttfamily,
                morecomment=[l][\color{magenta}]{\#}
}
\begin{lstlisting}
fn print(a: &Vec<i32>) {}

fn main() {
    let a = Vec::new();
    a.push(1);
    a.push(2);
    
    print(&a);
    print(&a);
}    
\end{lstlisting}

\end{frame}

% Slide 10
%\begin{frame}[fragile]
%\frametitle{Borrow in Rust}

%\begin{itemize}
%    \item Borrows values are valid for a lifetime
%\end{itemize}
    
%\lstset{language=Java,
%                basicstyle=\ttfamily,
%                keywordstyle=\color{blue}\ttfamily,
%                stringstyle=\color{red}\ttfamily,
%                commentstyle=\color{green}\ttfamily,
%                morecomment=[l][\color{magenta}]{\#}
%}
%\begin{lstlisting}
%fn print(a: &Vec<i32>) {}

%fn main() {
%    let a = Vec::new();
%    a.push(1);
%    a.push(2);
    
%    {
%    	let b = &a;
%    }
    
%    print(a);
%}    
%\end{lstlisting}

%\end{frame}

% Slide 11
\begin{frame}[fragile]
\frametitle{Mutable Borrow in Rust}

\begin{itemize}
    %\item Cannot have both shared and mutable reference at the same time
    \item There can only be one unique reference to a var. that is mutable
    \item Borrows values are valid for a lifetime
\end{itemize}
    
\lstset{language=Java,
                basicstyle=\ttfamily,
                keywordstyle=\color{blue}\ttfamily,
                stringstyle=\color{red}\ttfamily,
                commentstyle=\color{green}\ttfamily,
                morecomment=[l][\color{magenta}]{\#}
}
\begin{lstlisting}
fn muliply(vec: &mut Vec<i32>) {
    for e in vec.iter_mut() { *e *= 2; }
}

fn main() {
    let mut vec: Vec<i32> = vec![1, 2];
    {
        let mut vec2 = &mut vec;
        muliply(&mut vec2);
    }
    muliply(&mut vec);
}
\end{lstlisting}

\end{frame}

% Slide 12
\begin{frame}[fragile]
\frametitle{Concurrency in Rust}

\begin{itemize}
    \item Using ownership to prevent data races
    \item Parallelism is achieved at the granularity of an OS thread
    \item Safety is achieved by requiring that a 'move' owns captured variables
    %\begin{itemize}
    %    \item a move owns its environment
    %\end{itemize}
\end{itemize}

\lstset{language=Java,
                basicstyle=\ttfamily,
                keywordstyle=\color{blue}\ttfamily,
                stringstyle=\color{red}\ttfamily,
                commentstyle=\color{green}\ttfamily,
                morecomment=[l][\color{magenta}]{\#}
}
\begin{lstlisting}
fn main() {
    let mut a = Vec::new();
    // 'move' instructs the closure to move out of 
    // its environment
    thread::spawn(move || {
        a.push("foo");
    });
    a.push("bar"); // error: using a moved value
}
\end{lstlisting}
   
\end{frame}

% Slide 13
\begin{frame}[fragile]
\frametitle{Concurrency in Rust}

\begin{itemize}
    \item Threads can communicate with channels
\end{itemize}

\lstset{language=Java,
                basicstyle=\ttfamily,
                keywordstyle=\color{blue}\ttfamily,
                stringstyle=\color{red}\ttfamily,
                commentstyle=\color{green}\ttfamily,
                morecomment=[l][\color{magenta}]{\#}
}
\begin{lstlisting}
fn main() {
    let (tx, rx) = mpsc::channel();
    let x = Box::new(5); // allocate 5 on the heap
    
    thread::spawn(move || {
        let result = 5 + *x;
        tx.send(result);
    });
    
    let result = rx.recv().unwrap();
    println!("{}", result);
}
\end{lstlisting}

\end{frame}

% Slide 14
%\begin{frame}[fragile]
%\frametitle{Unsafe in Rust}
   
%\begin{itemize}
%    \item Unsafe blocks in Rust are used to bypass protections put in place by the compiler
%    \item Dereferencing raw pointers
%\end{itemize}

%\lstset{language=Java,
%                basicstyle=\ttfamily,
%                keywordstyle=\color{blue}\ttfamily,
%                stringstyle=\color{red}\ttfamily,
%                commentstyle=\color{green}\ttfamily,
%                morecomment=[l][\color{magenta}]{\#}
%}
%\begin{lstlisting}
%fn main() {
%    let raw_p: *const u32 = &10;

%    unsafe {
%        assert!(*raw_p == 10);
%    }
%}
%\end{lstlisting}

%\end{frame}

% Slide 15
\begin{frame}[fragile]
\frametitle{Cargo}

\lstset{language=Bash,
                basicstyle=\ttfamily,
                keywordstyle=\color{blue}\ttfamily,
                stringstyle=\color{red}\ttfamily,
                commentstyle=\color{green}\ttfamily,
                morecomment=[l][\color{magenta}]{\#}
}

\begin{itemize}
    \item Package manager, similar to pip in Python
    \item Download and manage dependencies
    \item Build the project
    \begin{itemize}
        \item Create Cargo.toml
        \begin{lstlisting}
$ cargo new hello_word --bin
        \end{lstlisting}
        \item Cargo.toml
        \begin{lstlisting}
[package]
name = "hello_world"
version = "0.1.0"
authors = ["Gerald Stanje"]
[dependencies]
regex = "0.1.33"
        \end{lstlisting}
        \item Build project
        \begin{lstlisting}
$ cargo build --release
Compiling hello_world v0.1.0 (file:///ho
me/geri/code/hello_world)
        \end{lstlisting}
    \end{itemize}
\end{itemize}

\end{frame}

\end{document}